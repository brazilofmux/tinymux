\documentclass[letterpaper]{book}
\title{TinyMUX 2.0 Manual}
\author{Stephen Dennis (AKA Brazil)}
\begin{document}
\maketitle
\chapter{Introduction}
\section{Philosophy of this Document}
There is no single author to this document --- only an editor. There is an
overarching structure to the document, but otherwise, changes by everyone are
welcome. Please do your part to make it better.

Things are first broken down into views or perspectives that cater to a
particular type of reader at a particular experience or skill level and look
at the subject under discussion from a particular angle. I ask that top-level
divisions be chosen that way. Sub-divisions from there are much more flexible.
Some redundancy between perspectives is unavoidable. With an index, this
redundancy can help to pull the document together.
\section{Let's Begin}
There seem to be two good ways to begin:
\begin{enumerate}
\item
Find an expert to answer all your questions and prompt you for the questions
you would be asking if you even knew where to start.
\item
Do some research on your own and learn by quiet trial-and-error.
\end{enumerate}
There are advantages and disadvantages to both approaches. I recommend that
you rely on yourself and use the experts sparingly in specific areas as a way
to accelerate yourself through those specific issues. Always be prepared and
ready to find your own way.

There are only so many experts and their expertise may be narrow. Even experts
in the same area are known to disagree --- in fact, usually these experts
aren't talking to each other unless they are talking about something they don't
know about yet or they are disagreeing. You have little way of knowing whether
someone is an expert until you are already all wrapped up in their advice and
the political baggage they may carry with them. You may even need to put
yourself into the awkward position of having to motivate one of these experts
to help you. Enough said about that.

Solitary reading and trial-and-error has its own problems. Documentation that
is available is both limited and out of date. In order to get a good idea of
what you're dealing, you may need to splice together information from several
sources. It is easy to get stuck on a small point and waste a lot of time.
Also, these game servers are large enough and complicated enough that
experimentation without the advantage of some experience or history can be
nearly impossible.
\chapter{New to TinyMUX}
\section{Frequently Asked Questions (FAQ)}
\subsection{What is a Text-based Game Server?}
It is a program that allows a multitude of people to connect to the same
text-based environment and interact with each other and with the environment.
The administrator runs this server on top of an operating system and the server
in turn allows other people to connect. % See Figure 1 below.

These layers are written in a particular computer language. The hardware
layers have their own C-like language for describing how a body of gates are
connected logically, geographical, thermally, etc.  Most operating systems are
written in C/C++ with some assembly language and some higher level scripting
or macro languages. This is also sometimes called hardcode.

All of the MUSH-type text-based game servers are written in C except MUX 2.0
which is a combination of C and C++. This is what people call hardcode.

All of the Softcode Global Packages are written in a mostly universal language
that doesn't have a name. You could call it MUSHcode, but then you are probably
referring specifically to MUSH 2.2.x or U1 or perhaps MUSH 3.0. You could call
it softcode, but then you are also probably including non-MUSH family languages
further away and more incompatible. For the purpose of this document, we'll
call it softcode with the understanding that we expect it to run on a particular
set of closely related servers and on TinyMUX 2.0 in particular.

\section{How to Begin Running TinyMUX}
The first thing you'll need is some server hardware with some flavor of Unix or
Windows running. It's more than a little helpful if this server has a dedicated
Internet or Intranet connection, as without it, you'll be using it by yourself.

You need to find the latest known-good distribution from
www.tinymux.com and download it. If you are using a web browser, it will
download it in 'binary' mode for you automatically, however, if you use an FTP
client manually, you'll need to be sure to put the transfer into binary. Even
between Unix machines, Non-binary transfers can be problematic.

Here is a typical manual FTP conversation:
\footnotesize{\texttt{
\begin{verse}
\$ \emph{ftp svdltd.com}\\
Connected to svdltd.com.\\
220 sdennis0 Microsoft FTP Service (Version 4.0).\\
User (svdltd.com:(none)): \emph{anonymous}\\
331 Anonymous access allowed, send identity (e-mail name) as password.\\
Password: \emph{sdennis@svdltd.com}\\
230-Welcome to Solid Vertical Domains, Ltd.\\
230 Anonymous user logged in.\\
ftp$>$ \emph{cd TinyMUX}\\
250 CWD command successful.\\
ftp$>$ \emph{binary}\\
200 Type set to I.\\
ftp$>$ \emph{ls *.tar.gz}\\
200 PORT command successful.\\
150 Opening ASCII mode data connection for file list.\\
miam.tar.gz\\
mux2.0-B1.tar.gz\\
mux2.0-B2.tar.gz\\
mux2.0-B3.tar.gz\\
mux2.0-B4.tar.gz\\
mux2.0-B5.tar.gz\\
mux2.0-B6.tar.gz\\
mux2.0-B7.tar.gz\\
226 Transfer complete.\\
139 bytes received in 0.03 seconds (4.48 Kbytes/sec)\\
ftp$>$ \emph{get mux2.0-B7.tar.gz}\\
200 PORT command successful.\\
150 Opening BINARY mode data connection for mux2.0-B7.tar.gz(655702 bytes).\\
226 Transfer complete.\\
655702 bytes received in 0.09 seconds (6975.55 Kbytes/sec)\\
ftp$>$ \emph{quit}\\
221 Goodbye from Solid Vertical Domains, Ltd.\\
\$
\end{verse}
}}
The suffixes on the end of the filename tell you which tools were used to
produce the file and in which order. Gzip handles '.gz' files, and tar handled
'.tar' files. So in order to unpack this distribution, we do the following:
\footnotesize{\texttt{
\begin{verse}
\$ \emph{gzip -d mux2.0-B7.tar.gz}\\
\$ \emph{tar xvf mux2.0-B7.tar}\\
\end{verse}
}}
\end{document}
